\documentclass[UTF8]{article}
\usepackage{ctex,geometry}
\geometry{a4paper,scale=0.83}
\title{文献阅读\ 文章一}
\author{徐培宾 201850050}
% \date{\today}

\begin{document}
    \maketitle
    \section{文章概述}
    \subsection{主要发现}
    准噶尔盆地中发现的中新世晚期的长颈鹿(獬豸)拥有特殊的头颈形态,这种形态极有可能与一种极端的性选择下的头部碰撞行为有关。

    \subsection{结论}
    长颈鹿之间激烈的性斗争不仅塑造了他们的头颈形态,而且促使他们能够占据边缘生态位并适应环境。

    \subsection{相关依据}
    \begin{enumerate}
        \item 有限元分析表明,獬豸与现存脊椎动物相比,其关节更适合高速的头部碰撞,其形态拥有最佳的适应能力。
        \item 獬豸的牙釉质同位素与此地区的其他物种有很大差。表明其饮食习惯与其他物种不同,占据的生态位不同。
    \end{enumerate}

    \section{我的观点}
    这篇论文为我提供了许多进化生物学研究方向的新角度。

    从发现上来说,对于新事物的研究可以从已有的生物标本入手进行分析,通过种属分析确定该生物的类群,研究其特征,研究人员抓住了其极具特色的头颈进行研究,从形态学角度入手,判断獬豸头颈形态的优势之处,并作出性选择的假设,通过其它方式验证这种假设。
    
    从方法上来说,其遵循假设检验的步骤,对于新发现的问题先提出假设,再结合物理、化学等多种方式进行分析,确定结论后又在进化方向进行了更深入的研究,通过研究其祖先,和分支物种,以及生态位,全面理解獬豸其特殊形态存在和产生的多种意义。在本文的方法中,最值得注意的是其完备的方法体系,包括利用物理模型分析头颈的特征,对獬豸的形态特征做初步判断,其后又将獬豸放在时间维度上与不同时代的长颈鹿分支进行比较,判断可能存在的进化方向上的差异,最后将獬豸放在同一时代的其他物种间进行比较,研究同一地区中獬豸生态位的变化。从特征、时空多角度分析了其形态存在的原因,并且三者互相印证,进一步证实了结论的正确性。
    
    从结论上来说,这个结论仍然不够完善,但就现有的研究结果来看,似乎已经能够很好的自证了。仅仅从单一物种分析中,我们只能得到这样局限于种内竞争的结论,随着更多化石证据的发掘,研究人员或许需要从种间竞争的角度重新思考这个问题,除了种内的性选择,是否有其他种外刺激促进了这一进化过程。
\end{document}