\documentclass[UTF8]{article}
\usepackage{ctex,geometry}
\geometry{a4paper,scale=0.83}
\title{藏族人群高原低氧适应遗传机制研究}
\author{徐培宾 201850050}
% \date{\today}

\begin{document}
    \maketitle
    \section{适应性遗传物质}
    藏族人群对于高原的低氧环境有着良好的适应能力,这一能力与其特殊的生理结构息息相关,
    分子学层面研究表明,EGLN1、EPAS1和PPARA是与藏族人高原适应最密切相关的三个基因。
    EPAS1基因4号内含子内的一个C至G的突变,该突变在藏族人群中的频率为87\%,而在汉族人群中仅为9\%。
    关联分析发现,该突变与藏族人群的Hb和红细胞数显著相关,但与血氧饱和度不相关。
    在藏族和汉族人群间的频率差异最大的98个SNP中,有25个和6个分别在EPAS1和EGLN1基因中。
    藏族人群中这两个基因所在的区域连锁不平衡程度比汉族人群强,
    而单体型的多态性却比汉族人群低,表明藏族人群中EPAS1和EGLN1基因各存在一个受到强烈选择作用的优势单体型。

    % \subsection{脑}
    % \subsubsection{脑皮层}
    % EGLN1的高原低氧适应性基因型主要与额叶、枕叶和岛叶皮层形态相关,在皮层和皮层下细胞中高表达, 
    % 可以调控突触密度和改变细胞迁移,并在中枢神经系统损伤后通过调节皮层神经元的突起延伸参与轴突重连。
    % 而EPASI和PPARA的适应性基因型主要影响舌回和颢叶。 EPAS1编码HF-2α,它可以促进缺血时星形胶质细胞的存活,
    % 抑制糖异生, 在血管重建中发挥重要作用。高原适应性遗传变异相关的脑区主要位于额叶、岛叶和视觉皮层。
    % 遗传适应性程度越高,左侧岛叶和枕叶外侧的皮层厚度就越厚,视觉皮层的表面积和体积就越大,
    % 高原人群以此更好地适应低氧环境,通过增加低氧敏感脑区神经和血管的发生来提高皮层低氧损伤抗性。

    % \subsubsection{脑白质}
    % 适应性研究中,EGLN1、EPAS1和PPARA表现出很强的高原正向选择性,这些选择性位点的基因突变与相对较低的血红蛋白水平有关。
    % EGLN1、EPAS1和PPARA在红细胞增殖、能量代谢、血管以及神经发生等广泛的生理反应中发挥作用,
    % 影响到白质纤维束的结构完整性、髓鞘化程度和走形分布等。研究表明,遗传适应程度越高,
    % 额叶、颞枕叶以及岛叶区白质纤维束的组织排布更加有序,纤维连贯性更好,
    % 纤维密度以及髓鞘化程度更高。脑白质的这些特点使高原人群在低氧适应过程中具有更强的低氧脑损伤保护性。

    % \subsection{呼吸系统}
    % 当平原人群移居高原暴露在低压低氧环境容易诱导红细胞生成,主要是由于HIF-2a的积累,易导致红细胞增多症。
    % 相比之下,藏族受到遗传保护不受红细胞增多症的影响。
    % 藏族富集的对低氧红细胞生成反应钝化相关的适应型基因型和其他增加代谢效率的序列突变一起降解HIF-2a。
    % 他们的EPAS1适应性序列突变导致在低氧条件下与野生型相比减少EPAS1的表达,钝化低氧反应的分子基础。

    \section{种族历史}
    遗传学研究都发现,绝大多数藏族人群的遗传组分都可追溯到新石器时代迁入青藏高原的东亚北方人群。
    考古学证据表明,青藏高原在晚期更新世(5至1万年前)时期已经有人类生存。
    青海东北部地区最早在6500年-5600年前开始有人类永久性居住,西藏最东部在5900年前有人类定居,
    而西藏中部的雅鲁藏布江河谷发现的永久居住遗址则可追溯到3750年前;
    这三个地区的海拔都在3000米以上,提示这些定居人群已经适应了高原低氧环境,并逐渐扩散到青藏高原各地。

    \section{遗传分子机制}
    利用全基因组分型与高通量测序等基因组学的方法,研究者能够在全基因组水平上扫描人类基因组中受到自然选择压力的位点或基因,
    利用全基因组SNP分型以及候选基因重测序的方法,研究者发现EPAS1是藏族人群中受到最强自然选择作用的基因,
    其他基因(如EGLN1、UCP2等)也存在分化,海拔越高,低氧选择压力越大。
    EGLN1、EPAS1和PPARA等基因表现出很强的高原正向选择性,这些选择性位点的基因突变与相对较低的血红蛋白水平有关,
    这些基因在红细胞增殖、能量代谢、血管以及神经发生等广泛的生理反应中发挥作用。

    \section{演化路径}
    通过群体遗传学的方法,我们能够统计得出藏族人群可能的演化途径,现代研究从全基因组分型数据入手,
    结合藏族人群的遗传结构、内部分化以及演化历史得出结论。藏族人群内部存在分化,且分化类型与藏语三大方言区的划分一致。
    藏族人群和汉族人群间有基因交流,且基因交流的方向是相互的。利用汉族人群和定日藏族人群作为现代藏族的两个祖先人群,
    推算得到现代藏族中汉族人群来源的遗传组分比例大于45.6\%。常染色体数据分析显示藏族和汉族人群的分离时间为2600-4600年前。
    ROH和IBD分析提示藏族人群历史中可能发生过瓶颈效应。藏族人群与北方汉族人群的遗传距离更近,为藏族人群起源于东亚北方人群提供了遗传学证据。

    \section{我的见解}
    藏族人群的形状演化存在从东亚北方人开始逐步演化的过程,受到地理环境因素的影响,演化原因主要是自然选择因素,
    逐步进化的藏族人群相比平原地区人群,由于在特定的基因上存在突变,更适合高原高海拔低氧高寒气候下的生存,并且在演化的过程中逐步进化,
    克服了红细胞增多症等平原地区人群在高海拔地区不适应的症状,适应型基因型和其他增加代谢效率的序列突变一起降解HIF-2a,
    减少EPAS1的表达,钝化低氧反应的分子基础。

    特殊的地理环境限制另一方面也加剧了藏族人群迁移后的瓶颈效应,一旦适合高原的少数个体产生,便能够近亲繁殖,逐渐固定高原的适应性基因。
    因此,其虽然存在于汉族人群以及东亚人中的相似性,但仍存在较大差异,经过近千年的演化之后,这种差异将变得更加显著,
    但随着近年来与汉族人群的交流日益频繁,两种基因型融合的可能将越来越高。

    从进化生物学的角度而言,藏族人群适应低氧的性状特征符合演化规律,从东亚北方人种开始,部分个体产生基因突变,
    迁入高原地区,后经过自然选择的筛选,部分强适应性个体存活,他们之间的基因相互结合,产生适应性更强的后代,
    由于遗传漂变的作用,该地区的基因受到这些奠基者的影响而住不形成与外界不同的基因组类型,而天然的地理屏障完美地保护了这一种族的发展。

    从研究的角度来说,探究复杂性状的产生可以先从遗传物质角度入手,构建系统进化树,比较物种间的相似关系,从中找到可能的进化路径,
    再结合物种的歉意历史,以及遗传差异产生的多种原因推测该性状的演化方式。

    \begin{thebibliography}{99}  
        \bibitem{ref1}郭直. 世居高原藏族人群脑对高海拔低氧的遗传适应[D]. 厦门大学,2020.
        \bibitem{ref2}张慧. 藏族人群高原适应关键基因EPAS1和EGLN1的功能研究[D]. 西藏大学,2017.
        \bibitem{ref3}张祥龙. 藏族人群高原低氧适应遗传机制及演化历史研究[D]. 中国科学院大学,2013.
        \bibitem{ref4}次仁潘多,央拉,尼玛玉珍,次仁桑珠,普布,巴桑卓玛.藏族登山运动员高原低氧生理效应与KIF1B基因的关联分析[J].高原科学研究,2019,3(01):95-101+121.DOI:10.16249/j.cnki.2096-4617.2019.01.012.
    \end{thebibliography}
\end{document}