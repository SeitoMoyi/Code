\documentclass[UTF8]{article}
\usepackage{ctex,geometry}
\geometry{a4paper,scale=0.8}
\title{浅探evolution和involution}
\author{徐培宾 201850050}
\date{\today}

\begin{document}
    \maketitle
    从英文词法上出发,involution和evolution两个词的区别在于前缀的不同,两者的前缀分别是in-和e-。我们不难在英语中找到这样相似的词,比如invade和evade,前者表示(向内的)入侵,后者则表示(向外的)躲避,由此可见这两个前缀存在动作方向上的区别。因此,involution强调的是(物种或群体向内的)挤压(导致群体的内耗加重),而evolution强调的是(物种或群体向外的)扩张(导致一个旧的物种被适应性更强的新物种取代)。

    虽然从词义上来说两者间有着相反的关系,但是从群体层面分析,两者在进化和淘汰上有着一定相似之处,不可否认的是,两者都能够将筛选出具有较高适应性的群体,evolution是在物种间进行选择,而involution则是在物种内进行选择。

    诚然,involution虽然在当下的社会环境中能够起到筛选和淘汰的作用,但是就长期来看,并不是一个物种evolution的最优解。过度的内耗使大量的精神物质财富白白浪费,非但没有推进人类社会的发展,甚至有可能反向阻碍社会进步,其本质是一种无视物质客观发展规律,强加于社会体系的谬论。好在,从长期来看,人类社会并不会始终停滞在当下的阶段,总是以当下的社会状况去衡量未来的人是不现实的,

    当下involution盛行其道的关键在于生产力与生产岗位的不平衡,当生产岗位的数量和生产力的数量不能达到平衡,处在生产岗位中的人就必须要将腾出一部分精力用于“保护”自身的岗位,而这部分精力便是无谓的内耗。这种“保护”的意识是一种示威,它会传播给他人危险的信号,当人们身处在这样一种环境中时,他们会感受到来自involution的焦虑情绪,久而久之,最后加入involution的行列之中,形成一场恶性循环。

    综上来看,我并不否认involution存在的价值,甚至我认为它可能是当下社会环境中的最优解。但是同时我也不认为这种错误的模式将会长期存在,一来是因为它不符合evolution的客观规律,将精力无意义地浪费在了低效率的一方面,加大了群体的内耗,而不是寻找向外寻找提升群体适合度的方法。二来是因为随着社会的发展,尤其是我国,将在未来拥有更加完善的社会福利体系与更多的工作岗位。保障好人民的生存安全,才是解决involution焦虑的根本方法。
\end{document}