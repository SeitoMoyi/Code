\documentclass[UTF8]{article}
\usepackage{ctex,geometry,graphicx,float,makecell,rotating,multirow,diagbox}
\geometry{a4paper,scale=0.8}
\title{中美关系中的博弈问题}
\author{201850050\ 徐培宾}
\date{\today}

\begin{document}
    \maketitle
    \section{背景}
    博弈论,由匈牙利数学家冯·诺依曼创立。博弈论研究的是,在两个或两个以上参与者的博弈中,各方为了自身利益最大化而研究对方的策略,并采取相应的有效对策,以冲突与合作的基本形式形成互动。

    博弈论能让人很自然地联想到棋类游戏,我们可以从最简单的双人棋局开始,了解博弈中普遍存在的基本形式和原理。一个简单的棋类游戏,参与者的数量是确定的,游戏规则是明确的,
    游戏目的是自身的利益最大化,游戏完全是通过策略(包括对方的策略错误)赢得的。因此,当我们将目光转向国际,国际间的斗争也正是一盘多人棋局,参与者数量是确定的,游戏规则是明确的,
    各方争夺的最终目标都是自己的利益。许多博弈论的学者早就意识到了这一点,他们将博弈论研究的成果应用到国际事务中去,极大地增加了博弈论的应用范围,也阐明了许多国际问题。
    博弈论给我们提供了一个新的视角观察国际局势,本文将一中美关系为例,以两个例子列举博弈论在中美博弈分析中的应用。

    \section{中美制造业博弈——囚徒困境}
    “囚徒困境”最早由普林斯顿大学数学家曾科于1950年提出。他用一个假想的囚徒故事反映了现实生活中一个极其深刻的事实,现在人们经常用它来分析经济和政治领域的各种现象。
    中美间的制造业竞争与合作的博弈无疑也是其中之一。

    改革开放早期,国家坚持引进外资的战略,给西方国家的投资者带来了很多的利好,于是纷纷在中国设厂,中国的制造业也逐渐兴起。如今,经过四十余年的发展,
    这些企业不少已经成为行业龙头企业,能够达到世界顶尖的制造水准,廉价的劳动力和广大的市场让欧美以及西方投资者都十分向往来中国投资,也正因为如此,中国成为当今世界上名副其实的第一制造业大国。

    中美博弈在制造业方面的博弈,是上游和下游的博弈。就美国而言,因为其高昂的人力成本,以及相对较少的劳动力,在传统制造业方面,不及人口多、成本低廉的中国制造业市场,
    但是,美国掌握着上游核心技术,负责制造业技术的研发与升级,许多国内的公司需要向欧美支付费用购买专利来学习先进的制造业技术,抑或是成为美国公司的代工厂,负责产品的组装这类简单且低回报的工作。
    不可否认的是,虽然我国在制造业上已经具备了一定的研发能力,但是总体而言还是逊于发达国家水平。尽管如今的中国拥有着世界一流的下游制造业基础,但是缺少决定性的上游技术,我们究竟是否应该与美国合作呢?

    中美双方对自己的制造业情况都有着清楚的认识。前些年,美国曾号召美国企业将制造业基地搬回美国本土,以实现利润的最大化,并且更好地保护自己的核心技术,应证了一句老话,
    “肥水不流外人田”。但是,美国撤回工厂的呼吁收效甚微,很明显,几乎所有的投资者都认为美国的工人不适合从事下游制造业,相同的工资可以在中国等地找到更好的下游制造业工厂。
    反观中国,中国的许多下有制造业代工厂也在积极地与国外大企业合作,尽管利润空间不大,但是能带来稳定的就业岗位和收获更多的技术支持。
    综上所述,中美间制造业上的竞争与博弈,双方信息对称,我们可以用囚徒困境来尝试分析这个问题。

    \begin{table}[H]
      \centering
      \caption{中美制造业博弈}
      \begin{tabular}{|l|c|c|}
        \hline
        \diagbox{美国}{中国}& \textbf{合作} & \textbf{不合作}\\
        \hline
        \textbf{合作} & (200,100) & (250,0)\\
        \hline
        \textbf{不合作} & (0,150) & (0,0)\\
        \hline
      \end{tabular}
    \end{table}
    \begin{itemize}
      \item 双方合作的情况下,美国获得高额上游利润200,中国则获得下游制造业利润100。
      \item 若美国合作,中国拒绝合作,中国能够在一定程度上制裁美国制造业,切断其下游供给,对自身有利,但是失去了重要的上游技术支撑,得益150。
      \item 若中国合作,美国拒绝合作,美国也能从一定程度上阻止技术流向中国,遏制中国制造业的发展,对自身有利,但是失去了联结的下游制造业工厂,获利250。
      \item 若双方均不合作,任何人无法从交易中获利,获利均为0。
    \end{itemize}

    接着我们用划线法选出合适的策略。

    \begin{table}[H]
      \centering
      \caption{划线法处理}
      \begin{tabular}{|l|c|c|}
        \hline
        \diagbox{美国}{中国}& \textbf{合作} & \textbf{不合作}\\
        \hline
        \textbf{合作} & (\underline{200},\underline{100}) & (\underline{250},0)\\
        \hline
        \textbf{不合作} & (0,\underline{150}) & (0,0)\\
        \hline
      \end{tabular}
    \end{table}

    可以看出,其中存在的均衡策略为(合作,合作),无论对方选择何种策略,选择合作都是最优的,这也不难解释中美双方制造业中间存在的分工与合作的问题。
    当双方都能够合作时,双方都能获得较大利益,双方总利益最大。当然,美国仍然独占着上游技术所得的大蛋糕,但是中国也能从中分一杯羹,对双方都有利的行为何乐而不为呢?

    美国在合作中收获饿更多的利益,毫无疑问,这是国家技术和实力所决定的,希望在未来,中国也能成为上游的技术出口国,不再受制于他人,只能拾取啃下来的一小块蛋糕。

    \section{中美科技博弈——先动优势与后动优势}

    1985年,罗宾逊和费内尔提出了先动优势的概念;后来,一些学者和研究者进行了更多的理论研究和实证分析,并取得了一些不错的研究成果。其中,解释先动优势有两个理论基础:
    经济理论基础——用进入壁垒理论和企业效用函数解释先动优势;行为理论的基础——主要运用顾客锁定理论和消费者偏好理论来解释先动优势。先动优势的来源主要包括:技术诀窍和技术领导力。
    先发企业的科技创新将为企业提供专有技术,从而确立企业的技术领先地位。产品和工艺创新中学习曲线的向下移动可以反映出先动企业将拥有更好的竞争优势;
    企业可以获得或建立声誉、品牌、企业文化、技术资源、专利、积累的知识和经验等无形资产;成本优势:先发优势有利于先开拓市场,提高市场占有率,降低产品成本,
    从而提高企业的经济效益,进一步推动企业开拓新市场。良性循环为企业的长远发展带来了更大的竞争优势。

    美国科技的腾飞早在第二次工业革命时期,比中国早了近百年,在此之后,其一直作为全球科技的领头羊,进入21世纪以后,美国曾经的技术优势变得愈发明显,
    当前,许多中国的工业产品开发仍然需要使用美国的底层逻辑编程、制作,尽管能制造出不错的科技产品,但归根结底,这些都已经“站在了巨人的肩膀上”,
    如果没有美国先行技术的打基础,我们不可能拥有如此便捷的开发过程。而这样的事件,只是美国统治全球科技的一个微不足道的缩影,
    如今美国能够站在超级大国的地位上对几乎所有国家发号施令,正是因为其有着科技的先动优势。

    美国的先动优势体现在多个方面。第一,美国是最重视科研开发的国家,通常人们会认为,美国已然坐稳全球科技霸主的交椅,即使削减科研支出也并不影响其地位,
    然而与事实相反的却是,美国的研发经费在逐年增加,并且在研发上的支出超过任何其他国家。这无疑是先动优势造成的,最先发展科技的美国也是最先认识到科技是国家实力中最重要的部分的,
    因此,它才会不惜花费重金大力支持研发事业,在中国还在迷茫之时率先在全民族的思想中根植创新意识,始终向着科技强国坚定迈进。常年的先行探索,让美国拥有一套完善的研发体制和高效的研发流程,
    促进其在研发方面能够提供相对于其他国家更多的支持,美国的技术突破有赖于政府、学术界和企业之间的“三角联盟”。每个联盟都有自己的部门优势。政府提供必要的公共物品,包括教育、
    基础设施和基础研究基金。大学和国家实验室在公共投资的支持下孵化和开发早期、高风险的想法和技术。然后,私营部门扩大了规模,将有希望的项目商业化,并自行开展了有价值的研究。

    更多的支持和更好的研发环境所带来的,不仅仅是美国人在创新方面的巨大优势,也吸引了众多海外精英们前往美国,先动优势的第二项内容,便是美国有着优秀的创新人才培育体制,能够吸引更多的人才。
    在体制的探索上,中国虽然已经在追赶,但是不可否认的是,美国利用其悠久的科研历史创造出一套适合全球精英的创新培养方法,这无疑也是经过长时间的检验得出的,
    中国现阶段的体制虽然已经有一定成效,但仍然需要经过更长时间的检验。数据表明,截至2016年,美国授予的科学和工程(S\&E)博士学位是所有国家中最多的。优秀的培养体制加上优秀的研发支持,
    使得这个老牌科技强国靠着自己的招牌吸收着一代又一代来自全世界各地的新鲜血液,形成研发创新的良性循环。

    第三,也是最重要的方面,美国独特的领导地位决定了它能够在自己擅长的科技领域制定规则,其通过制定规则的能力攫取巨额利益,让其他小国家只能遵守规则,从剩余的一点利益中争夺。
    毫无疑问,标准是技术创新的一个重要方面。标准规定了产品、程序和人员的设计和安全要素。标准化组织有助于确定使用谁的技术来获得高利润的许可和获胜公司的显著竞争优势,
    以及如何使用这些技术。在当下的大部分领域内,美国公司仍然在主导关键技术。当然,中国也在努力,中国在制定标准方面采取了更为国家主导的方式,通过政府授权机构更严格地控制标准制定,并支持行业积极参与标准制定团体。
    虽然短期内仍不可能冲破美国制定的行业标准,但随着时间的推移,越来越多的企业将有能力代表中国政府与美国在科技舞台上扳一扳手腕。

    第四,先动优势的美国企业为了能够在市场竞争中长期占据有利地位,在长时间的科研斗争中,开发出了一套属于自己的技术保密系统,并且正在日趋完善,如今的美国拥有全世界最完美的保密技术,也是拥有保密技术最多的国家,
    处于行业龙头的一些美国企业到达外国市场时,能够处于技术垄断地位,也正归功于此。中国处于后动劣势中,关键技术上始终被具有先动优势的美国“卡脖子”,一方面激发了中国研究者自身的研发潜力,
    另一方面却也更加凸显了美国先动优势的有利地位。调查数据显示,2017年至2020年2月,中金国际美国对中国相关交易的结清率约为60\%,从奥巴马执政期间超过95\%的比例下降。大多数在被拒绝的交易中,
    科技行业是被否决的,而非技术交易在奥巴马时代的水平上得到了批准。此后,国会通过了《出口管制改革法案》(ECRA),将出口限制扩大到对美国国家安全至关重要的“新兴”和“基础”技术。ECRA为出口管制的使用制定了指导原则,
    包括“只有在充分考虑到对美国经济的影响之后”才采取行动,并针对“能够被用来对美国构成严重国家安全威胁的核心技术和其他物品”进行国家安全控制。ECRA要求商务部通过工业和安全局(BIS)与私营部门协商,
    通过跨部门程序确定每一类产品。国际清算银行甚至公布了一份包括生物技术和人工智能在内的14种受管制新兴技术类别的广泛清单。由此看来,中国企业在研发方面仍然任重而道远,美国全方位保障自己的先动优势,
    无疑是为了更好地保存实力,并且压迫后来的国家,发展中的中国需要扛住后动劣势,集中力量,冲破美国的打压。

    面对美国单方面的科技威胁,中国的研发者也并没有坐以待毙,中国正将后动优势运用到自身的发展之中,努力追赶美国的脚步。

    后动优势是指相对于行业的先进入企业,后进入者由于较晚进入行业而获得的较先动企业不具有的竞争优势,通过观察先动者的行动及效果来减少自身面临的不确定性而采取相应行动,获得更多的市场份额。
    有学者对后动优势进行了研究分析,他们把后动优势的来源归结为以下四个方面:第一,具有开拓性的先动者地位比后动者地位所要付出的成本和代价要大得多,而且先动者几乎没有获得什么经验曲线效应。
    第二,由于技术变革速度很快,早期投资的设备和技术会很快过时,而后动者可以采用最先进的技术和设备。第三,由于顾客对先动者的忠诚度很弱,后动者很容易就能打开市场。
    第四,先动者付出巨大代价获得的技术和经验可能轻易地被模仿甚至超过。对于国家而言,后动优势主要体现在两个方面:其一,对于后发国家而言,先发国家拥有更先进的技术,
    因此如果后发国家能够很好地引进发达国家的技术,先发国家花费很长时间实现的经济增长,后发国家在短时间内就可以实现。其二,后发国家可以借鉴先发国家在工业化过程中出现的经验教训,
    避免不必要的弯路,从而实现弯道超车。技术不断的革新和产业不断的升级,正是现代经济快速增长的决定性因素,对发达国家和展中国家来说都是如此。
    但是,发达国家和发展中国家稍有不同,发达国家在18世纪的工业革命以后,一直处于全世界的技术和产业的最前线,任何技术革新和产业升级都必须自己研发,但是研发投入非常大,风险非常高。
    发展中国家使用的技术和现有产业都在世界技术和产业链的内部,其技术革新和产业升级可以通过模仿、导入、集成来实现,其成本和风险远低于自己的研究开发。这就是所谓的后发优势。

    对于中国而言,一方面,欧美的先进经验在短时间内极大地促进了我国的发展,技术的引进也深刻地影响了中国的发展速度。另一方面,正因为有美国作为先驱者始终走在科技前沿,中国才能够观察美国行动上的失败之处,
    总结经验和教训,少走弯路,实现更快的赶超和长期以来的高速发展。未来中国将逐渐从模仿转变为创新,在多个科技领域中实现领跑。我们也欣喜地看到,中国在加大科研投入,吸引海外人才,
    更多地参与国际科技标准的制定,形成科技发展的良性循环,也许在未来,我们将能够培育出更多本土的科技人才,成为全球科技的最大支柱。
    综上所述,即使中国失去了先动优势,仍然能够灵活运用后动者的立场为自己制造优势,快速提升自身的竞争力,如果能够利用对手的弱点为自身制造优势,将会更好地向发达国家迈进。

    \section{结语}
    中美两国间的博弈,显然不是一朝一夕形成的,亦不可能在一朝一夕间结束,换言之,中美两国间的博弈是一场重复博弈、无限次数博弈,在这样的博弈中存在着许多更小个体间的子博弈,是一个复杂的博弈模型,
    文中所提到部分仅能从微观和宏观两个角度一探其究竟,并不能完全代表中美间的所有博弈。但是可以明确的是,当今的国际关系中,合作所带来的两方收益之和往往是最大的,鉴于这是一场重复博弈,我们就更应该以合作为重,
    将目光放长远,不能只考虑眼前利益,否则,两方中的任何一方都将会因为选择背叛而付出承重的代价。正如时代所呼吁我们的那样,和平与发展才是这个时代的主旋律,争夺利益的斗争固然是不可避免的,
    但是从博弈论的角度分析这场重复博弈,真正笑到最后的必然是理性且懂得合作的一方。
\end{document}