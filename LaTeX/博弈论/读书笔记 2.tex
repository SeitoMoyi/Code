\documentclass[UTF8]{article}
\usepackage{ctex,geometry,graphicx,float,makecell,rotating,multirow,diagbox}
\geometry{a4paper,scale=0.8}
\title{读书笔记 2 \ 从原始走向阶级}
\author{201850050\ 徐培宾}
\date{2022年5月22日}

\begin{document}
    \maketitle

    在第一部分中,卢梭详细讲述了人类从原始状态逐渐进化的过程,理想中的原始人没有邪恶的情感,保留着善良朴实的内心,他们自给自足,保护自己和生存下去是他们追求的两大目标。而第二部分中,作者意图明显地将“社会”的概念引入其中,有别于原始人的社会,这样的社会更富符合现代的社会体系,它建立在财富的基础上,将人与人之间的不平等逐渐拉开,或许,正是因为人们的生活逐渐走向富足时,才使得其更加在意自己在氏族与社会中的个人地位起来,而不用去担忧生存的问题。

    作者在一开始便点明了这些活动的动机,并将其称之为“对幸福的追求”,尽管从一定程度上而言,这样的称呼并无问题,但是考虑到这些“追求”常常转化为恶毒的想法,我仍旧对这样的称谓不敢苟同,依我看来,或许称其为满足自身利益的欲望更加合适。从这个角度来看,原始人在最开始意识到对自己财产的时候,便是为了保障自身的利益,他们将土地划分,牛羊私有,,为了向他人宣示自己的利益。当这样的行为变得随处可见,人们便会意识到自己与他人的财产存在着区别,有人的土地优于自己,有人的家畜多余自己,当意识到这点以后,人们不再追求“幸福‘,而是为了满足自身获取财富的利益而行动。接着,当一部分人逐渐壮大起来后,贪婪的欲望会让他们将手伸向所有人,这便是征服的诱惑。

    这些崛起的富人成为统治者似乎是理所当然的,一切顺理成章地进行,人民或许会有反抗,但是富人们有足够的财力去对付这些造反的被统治者们,最后,统治者们搬出冠冕堂皇的法律来将这种统治阶级与被统治积极间的关系永远固定下来,这些文书法律就像是一把枷锁,牢牢地扼住被统治阶级的咽喉,一旦被统治者适应了没有自由的生活后,他们便不再渴望自由,以至于世袭制君主的诞生,他们也不闻不问,他们不希望改变“和平美好“的生活方式,即使活在统治之下。这无疑是符合历史的观点的,当一个伟大的王朝等级建立后,百年内几乎不会有人起兵造反,而动乱年代往往是小国割据,群雄争霸的。只因为统治者之间的对立造成了被统治阶级惶惶不安,担心财产的损失和美好生活的离去。

    章节的末尾,卢梭聊到了君主与法律,他认为,如果人民都遵守着原始的状态,法律便没有必要诞生,但是他显然回避了问题最重要的部分,这也正是我认为的本书中观点最为不堪的一部分。他没有能够坚定起君主起到的统治作用实质上是对人民的剥削这一观点,反而开始吹捧赞美法国的君主制度与合理的法律条文。诚然,他为了能够获得君主的认可,撰写了这篇文章,但是这样一番言论反过来推翻了其之前的观点,他认为“以被统治者的智力无法发现法律存在的剥削被统治阶级的漏洞,只有统治者才能利用法律“,在我看来,法律是由人制定的,更确切的说,是由统治阶级制定的,它也与人一样并不完美,并不能盲目得肯定某部法律,不完美的法律必然会存在得益者和负得益者,或许在今天,研究如何利用法律将人类平等约束,比起让人类倒退回到原始社会,会更有意义。

\end{document}