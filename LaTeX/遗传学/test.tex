\documentclass[UTF8]{article}
\usepackage{ctex,geometry,graphicx,float,makecell,rotating,multirow,diagbox}
\geometry{a4paper,scale=0.8}
\title{实验二\ 基因分离定律和自由组合定律的验证}
\author{201850050\ 徐培宾}
\date{2022年4月20日}

\begin{document}
    \maketitle
    \section{实验目的}
    通过实验验证分离规律和自由组合规律,掌握果蝇杂交的实验技术,在实验中熟练运用生物统计的方法对实验数据进行分析。
    \section{实验原理}
    \subsection{基因分离定律}
    一对等位基因在杂合状态中保持相对的独立性,在配子形成时,按原样分离到不同的配子中去,理论上配子分离比是1:1,1F2代基因型分离比是1:2:1,若显性完全,F2代表型分离比是3:1。
    
    控制体色性状的突变基因位于2号常染色体,正常体色对黑体完全显性,用正常体色果蝇与黑体果蝇交配,得到F1代都是正常体色,F1代雌雄个体之间相互交配,F2代产生性状分离,出现两种表现型。
    \subsection{基因自由组合定律}
    不同相对性状的等位基因在配子形成过程中,等位基因间的分离和组合是互不十扰,各白独立分配到配子中去,它们所决定的两对相对性状在2代是白由组合的,在杂种第二代表型分离比就呈9:3:3:1。

    控制体色性状的突变基因位于2号常染色体,正常体色对黑体完全显性,控制眼色性状的突变基因位于性染色体。红眼对白眼完全显性,用黑体红眼果蝇(雌)与正常体色白眼果蝇(雄)交配,得到F1代都是正常体色,F1代雌雄个体之间相互交配,F2代产生性状分离,出现四种表现型。
    \section{实验材料}
    \begin{itemize}
        \item 用具:显微镜,麻醉瓶,培养瓶,滤纸,毛笔,标签,恒温培养箱
        \item 材料:野生型果蝇原种、双隐(黑体、残翅)突变型果蝇原种
        \item 药品:乙醚,乙醇,培养基
    \end{itemize}
    \section{实验操作}
    \subsection{设计杂交组合}
    亲本:正交(野生雌X双隐雄),反交(双隐雌X野生雄)

    正交产生的F1代继续杂交,产生F2代。
    \subsection{杂交接种}
    \begin{enumerate}
      \item 清除成蝇(生长好、接近羽化蛹多者)
      \item 清除成蝇后每间隔10小时重复收集处女蝇。
      \item 杂交接种(次日就要检查,如有死亡及时补救)。杂交瓶上注明:杂交组合、实验日期、实验者姓名。
      \item 部分蛹变黑,F1即将孵出前,移出亲本蝇。亲本蝇冻存(为防止失败,可留存活蝇适当时间),配制新培养基,用于F1同胞交配。
      \item F1代果蝇的观察和交配。F1代果蝇孵出7-9d观察统计F1,并选5对再杂交于一个或多个新瓶。
      \item F1杂交一周后,移出F1代果蝇。
      \item 第一只F2代果蝇出现后10天内,观察、统计F2代的结果,进行分析。
    \end{enumerate}
    \section{实验结果与分析}
    \subsection{杂交结果}
    \begin{table}[H]
      \begin{center}
        \caption{反交(双隐雌X野生雄)}
        \begin{tabular}{|l|c|c|}
          \hline
            & \textbf{雄} & \textbf{雌}\\
          \hline
          灰体长翅 & 17 & 9\\
          灰体残翅 & 8 & 6\\
          黑体长翅 & 7 & 3\\
          黑体残翅 & 0 & 3\\
          \hline
          $\chi^2$检验 & \multicolumn{2}{|c|}{2.178}\\
          \hline
        \end{tabular}
      \end{center}
    \end{table}
    \begin{table}[H]
      \begin{center}
        \caption{正交(野生雌X双隐雄)}
        \begin{tabular}{|l|c|c|}
          \hline
            & \textbf{雄} & \textbf{雌}\\
          \hline
          灰体长翅 & 30 & 26\\
          灰体残翅 & 3 & 5\\
          黑体长翅 & 4 & 8\\
          黑体残翅 & 1 & 5\\
          \hline
          $\chi^2$检验 & \multicolumn{2}{|c|}{6.542}\\
          \hline
        \end{tabular}
      \end{center}
    \end{table}
    从表中可以看出,数据虽然呈现出比较明显的大小差异,但是不能体现出9:3:3:1的分离比,可能是由于果蝇数量少,存在误差。

    通过$\chi^2$检验可以发现,正交数据差异较为明显。

    此外,不同性状的果蝇性别差异较大,可能原因有两个,一是总量过小,存在一定的误差,二是收集在F3产生时停止,可能有一部分F2未计入数据,导致F2的统计数据不正确。
    \begin{table}[H]
      \begin{center}
        \caption{反交(双隐雌X野生雄)}
        \begin{tabular}{|l|c|c|}
          \hline
            & \textbf{雄} & \textbf{雌}\\
          \hline
          灰体 & 25 & 15\\
          黑体 & 7 & 6\\
          \hline
          $\chi^2$检验 & \multicolumn{2}{|c|}{6.289$\times 10^{-3}$}\\
          \hline
          长翅 & 24 & 12\\
          残翅 & 8 & 9\\
          \hline
          $\chi^2$检验 & \multicolumn{2}{|c|}{114533}\\
          \hline
        \end{tabular}
      \end{center}
    \end{table}
    \begin{table}[H]
      \begin{center}
        \caption{正交(野生雌X双隐雄)}
        \begin{tabular}{|l|c|c|}
          \hline
            & \textbf{雄} & \textbf{雌}\\
          \hline
          灰体 & 33 & 31\\
          黑体 & 5 & 13\\
          \hline
          $\chi^2$检验 & \multicolumn{2}{|c|}{0.406}\\
          \hline
          长翅 & 34 & 34\\
          残翅 & 4 & 10\\
          \hline
          $\chi^2$检验 & \multicolumn{2}{|c|}{4111}\\
          \hline
        \end{tabular}
      \end{center}
    \end{table}
    从不同性状的统计来看,显性性状和隐性性状之间均可以看出近似3:1的特点,通过$\chi^2$检验可以发现,灰体和黑体的分离比比长翅和残翅的分离比更明显。
    \section{总结与讨论}
    \begin{enumerate}
        \item {
          \textbf{实验中的注意事项:}
          \begin{itemize}
            \item 果蝇要麻醉要适度,用毛笔等温和的工具挑取果蝇,以免对果蝇造成伤害;
            \item 果蝇麻醉后应及时观察,以免果蝇苏醒后飞走;
            \item 培养基干湿度适中,防止果蝇粘在培养基上,麻醉后的果蝇应轻轻倒回培养瓶,防止沾上培养基;
            \item 观察性状时要仔细,特别是黑体和灰体的区分要明确;
            \item 及时收集处女蝇,避免处女蝇在杂交实验前已交配;
            \item 分工合作,高效地进行实验和数据统计。
          \end{itemize}
        }
        \item {
          \textbf{为什么长翅-残翅的分离比不明显?}

          可能原因一是长翅果蝇在收集时飞走了两只,对实验数据产生了一定影响。

          二是因为残翅果蝇不能飞行,比较脆弱,可能有部分个体压在培养基中死亡,没有计入数据中。

          三是收集在F3产生时停止,可能有一部分F2未计入数据,导致F2的统计数据不正确。
        }
        \item {
          \textbf{果蝇杂交实验的改良?}
          \begin{itemize}
            \item 有条件的情况下,选用更多的果蝇进行饲养,避免因为基数少,造成明显的实验误差。
            \item 选取更好的果蝇收集方式,如可以用离心管收集处女赢得用,方便保存并且避免频繁地进入实验室收集\textsuperscript{\cite{ref1}}。
            \item 长时间收集F2,在确保F2收集的雌性全是处女蝇的前提下,收集并处死每次产生的F2,防止F3产生,延长收集时间直至F2手机完全。
          \end{itemize}
        }
    \end{enumerate}
    \begin{thebibliography}{99}  
        \bibitem{ref1}张远莉,庞延军.果蝇杂交实验中处女蝇收集方法的改进[J].生物学教学,2018,43(12):56-57.
    \end{thebibliography}
\end{document}