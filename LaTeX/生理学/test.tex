\documentclass[UTF8]{article}
\usepackage{ctex,geometry,graphicx,float,makecell,rotating,multirow,diagbox}
\geometry{a4paper,scale=0.8}
\title{实验四\ 心脏期外收缩、代偿间歇与

心脏的容积导体特征、心脏的电活动与机械运动的关系}
\author{3A组\ 徐培宾}
\date{\today}

\begin{document}
    \maketitle
    \section{实验目的}
    \begin{enumerate}
    \item 学习暴露蛙类心脏的方法和蛙类心搏曲线的描记方法。
    \item 熟悉蛙心的结构和各部分的自动节律性活动。
    \item 观察描记蛙心的期前收缩与代偿间歇。
    \item 学习在体、离体蛙心电图的记录方法及容积导电规律。
    \end{enumerate}
    \section{实验原理}
    \subsection{蛙类心脏}
    蛙类心脏有两个心房和一个心室,心脏活动具有自律性,蛙心脏的起搏点是静脉窦。心房、心室的自律细胞称为潜在起搏点。静脉窦的节律最高,心房次之,心室最低。
    正常情况下,心脏的活动节律受控于静脉窦的节律,其活动顺序为:静脉窦、心房、心室。如果静脉窦的兴奋传导受阻,潜在起搏点可取代静脉窦引发心房、心室收缩活动。
    静脉窦、心房、心室的收缩活动可以通过张力传感器描记下来。

    自律细胞的兴奋性和节律与许多因素有关,其中体温是影响兴奋性的重要因素。分别单独加温心室、心房、静脉窦,观察其节律变化,可以证明三者节律性的高低。

    心肌的技能之一是具有较长的不应期,绝对不应期几乎占整个收缩期,有效不应期几乎相当于心室整个收缩期及舒张期的前三分之一。在心室收缩期内给予任何刺激,心室都不发生反应。
    在心室的舒张期特别是舒张的中后期,给予心室一个闽上刺激,则引发一次正常自律性以外的收缩反应,成为期前收缩或期外收缩。
    期外收缩本身也有一个有效不应期,当静脉窦传来的正常节律性兴奋刚好落在这个不应期内时,心室不再发生收缩反应,须待静脉窦传来下一次兴奋才能发生收缩反应。
    因此,在期外收缩之后,就会出现一个较长于正常节律性活动的间歇期,称为代偿间歇。

    心肌组织的这个特点,使得心肌不会像骨骼肌那样产生完全强直收缩,保证心肌始终作收缩与舒张的节律性活动,从而使心脏有血液回心充盈的时期,实现心脏的泵血功能。
    % \begin{figure}[H]
    %   \centering
    %   \includegraphics[width = 12cm]{heart.png}
    %   \caption{蛙类心脏}
    % \end{figure}
    \subsection{心脏的容积导体特征}
    由于机体任何组织与器官都处于组织液的包围之中,而组织液作为导电性能良好的容积导体,可将组织和器官活动时所产生的生物电变化传至体表。
    故在体表或容积导体中的远隔部位可记录出某一器官或组织活动的电变化,如心脏活动所产生的生物电变化,可通过引导电极置于体表的不同部位记录下来,即心电图。
    \section{实验材料}
    \begin{itemize}
        \item 用具:PowerLab实验系统、常用手术器械、培养皿、蛙心夹、蛙钉、蛙板、棉线、张力换能器、插入电极、滴管、培养皿、小试管、冰块、热水、支架、秒表、木夹等
        \item 材料:蛙
        \item 药品:任氏液
    \end{itemize}
    \section{实验操作}
    \subsection{暴露心脏}
    取蟾蜍一只,双毁髓后背位固定于蛙板上。左手持手术镊提起胸骨下方的皮肤,右手持金冠剪剪开一个小口,然后将剪刀由开口处伸入皮下,向左、右两侧下颌角方向剪开皮肤。
    将皮肤掀向头端,再用手术镊提起胸骨下方的腹肌,在腹肌上剪一口,将金冠剪紧贴胸壁伸入胸腔(勿伤及心脏和血管),沿皮肤切口方向剪开胸壁,剪断左右乌喙骨和锁骨,
    使创口呈一倒三角形。左手持眼科镊,提起心包膜。右手用眼科剪刀剪开心包膜,暴露心脏。
    \subsection{观察心脏的结构}
    一个心室,其上方有两个心房,房室之间有房室沟。心室右上方有一动脉圆锥,是动脉根部的膨大。动脉干向上分成左右两分支。用蛙心夹夹住少许心尖部肌肉,
    轻轻提起蛙心夹,将心脏倒吊,可以看到心脏背面有节律搏动的静脉窦。在心房与静脉窦之间有一条白色半月形界线,称为窦房沟。前、后腔静脉与左、右肝静脉的血液流入静脉窦。
    \subsection{记录在体蟾蜍心波动和蛙心电图}
    \begin{enumerate}
      \item 将心电图测试电极分别与蟾蜍的两上肢相连(右上-,左上+),参比电极与蟾蜍的右下肢相连。记录蟾蜍正常心电图。数字滤波notch 50 Hz。
      \item 以蛙心夹固定心尖部位,使心脏离开胸腔,呈30度角与张力换能器相连。确认张力换能器所在通道 。进入chart软件,将记录速度为2k/s,张力记录通道的低通过设置为10Hz, 灵敏度10mV左右。同时记录蟾蜍心搏曲线和心电图。
      \item 调整张力换能器和滑轮位置,使蟾蜍心脏离开胸腔,呈45度角与张力换能器相连。记录蟾蜍心搏曲线和心电图。
      \item 调整张力换能器和滑轮位置,使蟾蜍心脏离开胸腔,呈60度角与张力换能器相连。记录蟾蜍心搏曲线和心电图。
    \end{enumerate}
    % \begin{figure}[H]
    %   \centering
    %   \includegraphics[width = 12cm]{helec.png}
    %   \caption{心肌细胞动作电位与心电图的关系}
    % \end{figure}
    \subsection{记录期外收缩和代偿间歇}
    \begin{enumerate}
      \item 以蛙心夹固定心尖部位,使心脏离开胸腔,呈30-45度角与张力换能器相连。确认张力换能器所在通道。
      \item 进入chart软件,将记录速度为2 k/s,张力记录通道的低通过设置为10 Hz,灵敏度10 mV左右。
      \item 设置“setup”-“stimulator”,将频率设置在1 Hz,将电流强度设置在1 mA,将“pulse” 设置为0.1 ms。将“stimulator mark”选项的内容也设置到张力记录通道。
      \item 以刺激器刺激心脏表面,连续记录,找到典型的期外收缩和代偿间歇,保存图片。
    \end{enumerate}
    \subsection{观察心搏过程}
    \begin{enumerate}
      \item 在心脏正常收缩的情况下,记录下静脉窦、心房和心室收缩15次所需的时间。
      \item 用玻璃离心管分别装一定量的冷水和热水,将底端抵在静脉窦处数分钟,记录此时心房、心室和窦房节分别收缩15次所需的时间。
    \end{enumerate}
    \subsection{记录离体心脏心电图}
    \begin{enumerate}
      \item 准备好含温热任氏液的平皿。放松心脏连线,提起蛙心夹,用组织剪取下完整心脏(注意包括静脉窦)。
      \item 将离体蟾蜍心脏置于盛有任氏液(30℃),的培养皿中,任氏液占培养皿的2/3。
      \item 将三个电极夹在培养皿壁上,每个电极隔120度角,将心脏放在培养皿的正中,心尖偏离电极参比电极逆时针45度角,记录心电图。
      \item 改变心脏心尖位置,记录不同位置的心电图。      
    \end{enumerate}
    % \begin{figure}[H]
    %   \centering
    %   \includegraphics[width = 12cm]{ecg.png}
    %   \caption{离体心脏心电图}
    % \end{figure}
    \subsection{组织分割}
    用眼科剪仔细地沿窦房沟、房室沟将静脉窦、心房、心室三部分剪开,放置在温热任氏液小皿中,计数其搏动15次的时间。
    \section{实验结果与分析}
    \subsection{观察心脏的结构}
    % \begin{figure}[H]
    %   \centering
    %   \includegraphics[height = 10cm]{heart1.jpg}
    %   \caption{心脏胸面观察}
    % \end{figure}
    % \begin{figure}[H]
    %   \centering
    %   \includegraphics[width = 10cm]{heart2.jpg}
    %   \caption{心脏背面观察}
    % \end{figure}
    % 一心室两心房结构清晰,动脉处有动脉圆锥,静脉膨大,静脉窦位于背面。
    % \subsection{记录在体蟾蜍心波动和蛙心电图}
    % \begin{figure}[H]
    %   \centering
    %   \begin{minipage}[t]{0.48\textwidth}
    %     \centering
    %     \includegraphics[width=8cm]{30d.png}
    %   \end{minipage}
    %   \begin{minipage}[t]{0.48\textwidth}
    %     \centering
    %     \includegraphics[width=8cm]{30d-1.png}
    %   \end{minipage}
    %   \caption{30$^{\circ}$角相连}
    % \end{figure}
    % \begin{figure}[H]
    %   \centering
    %   \begin{minipage}[t]{0.48\textwidth}
    %     \centering
    %     \includegraphics[width=8cm]{45d.png}
    %   \end{minipage}
    %   \begin{minipage}[t]{0.48\textwidth}
    %     \centering
    %     \includegraphics[width=8cm]{45d-1.png}
    %   \end{minipage}
    %   \caption{45$^{\circ}$角相连}
    % \end{figure}
    % 随着心脏角度扩大,心脏离体距离变大,血液压力变化;心脏暴露于空气中,同时影响体内电流。

    % 45度角相连时出现了高尖P波,其通常是右心房扩大的象征。由于高尖P波多由肺动脉高压或先天性心脏病引起,也称为肺型P波或先天性P波。
    % \begin{figure}[H]
    %   \centering
    %   \begin{minipage}[t]{0.48\textwidth}
    %     \centering
    %     \includegraphics[width=8cm]{60d.png}
    %   \end{minipage}
    %   \begin{minipage}[t]{0.48\textwidth}
    %     \centering
    %     \includegraphics[width=8cm]{60d-1.png}
    %   \end{minipage}
    %   \caption{60$^{\circ}$角相连}
    % \end{figure}
    % 60度角时,高尖P波消失,但P波仍呈现不规则的双峰状。
    % \subsection{记录期外收缩和代偿间歇}
    % 刺激无反应,调节刺激强度和频率大小均无效果。
    % \subsection{观察心搏过程}
    % \begin{table}[H]
    %   \begin{center}
    %     \caption{心搏次数观察}
    %     \begin{tabular}{|l|c|}
    %       \hline
    %       \textbf{温度} & \textbf{心搏次数/分钟}\\
    %       \hline
    %       室温 & 60\\
    %       冰水处理 & 43\\
    %       热水处理 & 67\\
    %       \hline
    %     \end{tabular}
    %   \end{center}
    % \end{table}
    % 虽然蛙是变温两栖动物,但是在不同温度下心率仍会发生变化,当静脉窦被冰水处理后,心率下降明显,而当热水处理后,心率稍稍上升。
    % 实验说明温度对心率的影响可通过直接作用于心脏的窦房结细胞而实现。在低温时,除了有温度对窦房结的直接作用外,迷走神经对心脏的支配作用增强也参与调节。\textsuperscript{\cite{ref1}}

    % 可能的原因是:当温度升高时,细胞代谢活动加强,供氧量增加,导致心率加快,当温度降低时,两栖类由于特有的冬眠特性,会在温度过低时产生较明显的心率下降。

    % 两栖类缺乏有效的温度调节功能,其体温与外界环境的温度相差无几。一般当环境温度降到15$^{\circ}$C以下或相对湿度降到80%以下时,多数中华蟾蜍个体发生越冬迁移(黎道洪,1987)。
    % 在冬眠期不食不动,其新陈代谢降到最低限度,乃至呈现麻痹状态,仅靠肝脏及脂肪体中储存的养分来维持生命(崔春月等,2007)。\textsuperscript{\cite{ref2}}
    % \subsection{记录离体心脏心电图}
    % \begin{figure}[H]
    %   \centering
    %   \begin{minipage}[t]{0.48\textwidth}
    %     \centering
    %     \includegraphics[width=8cm]{0.png}
    %   \end{minipage}
    %   \begin{minipage}[t]{0.48\textwidth}
    %     \centering
    %     \includegraphics[width=8cm]{00.png}
    %   \end{minipage}
    %   \caption{心尖0$^{\circ}$角放置}
    % \end{figure}
    % \begin{figure}[H]
    %   \centering
    %   \begin{minipage}[t]{0.48\textwidth}
    %     \centering
    %     \includegraphics[width=8cm]{180.png}
    %   \end{minipage}
    %   \begin{minipage}[t]{0.48\textwidth}
    %     \centering
    %     \includegraphics[width=8cm]{1800.png}
    %   \end{minipage}
    %   \caption{心尖倒置放置}
    % \end{figure}
    % 上述两种的心尖摆放恰好相反,其心电图也恰好相反。
    % \begin{figure}[H]
    %   \centering
    %   \begin{minipage}[t]{0.48\textwidth}
    %     \centering
    %     \includegraphics[width=8cm]{n45.png}
    %   \end{minipage}
    %   \begin{minipage}[t]{0.48\textwidth}
    %     \centering
    %     \includegraphics[width=8cm]{n450.png}
    %   \end{minipage}
    %   \caption{心尖向负(N)极偏45$^{\circ}$角放置}
    % \end{figure}
    % \begin{figure}[H]
    %   \centering
    %   \begin{minipage}[t]{0.48\textwidth}
    %     \centering
    %     \includegraphics[width=8cm]{p45.png}
    %   \end{minipage}
    %   \begin{minipage}[t]{0.48\textwidth}
    %     \centering
    %     \includegraphics[width=8cm]{p450.png}
    %   \end{minipage}
    %   \caption{心尖向正(P)极偏45$^{\circ}$角放置}
    % \end{figure}
    上述两种的心尖摆放关于参考电极所在的中线对称,其心电图也相反。
    \subsection{组织分割}
    \begin{table}[H]
      \begin{center}
        \caption{离体组织心搏次数观察}
        \begin{tabular}{|l|c|}
          \hline
          \textbf{组织} & \textbf{心搏次数/分钟}\\
          \hline
          静脉窦 & 46\\
          心房 & 40\\
          心室 & 未观察到\\
          \hline
        \end{tabular}
      \end{center}
    \end{table}
    心室部分没有观察到节律,原因一是剪下的心室部分并不完整,可能破坏了原有心室组织的结构,导致跳动不明显;二是等待时间不够长,可能没有恢复其本身节律。

    静脉窦相比在体时速率有所减慢,推测可能原因:一是离体后血液供氧不足,心率降低;二是心脏离体久置后活性下降,心率降低。
    \section{总结与讨论}
    \begin{enumerate}
        \item {
          \textbf{实验中的注意事项:}
          \begin{itemize}
            \item 彻底双毁髓,以免实验中肌电干扰;
            \item 把蟾蜍按正常体位,腹侧向上钉在蛙盘上;
            \item 应剪开蟾蜍四肢的皮肤,暴露出肌肉,将测试电极夹在肌肉上;
            \item 剪开胸腔创口时不要太大,尽量不要暴露肺和肝脏,剪胸骨和肌肉时紧贴胸壁剪刀尖上挑,以免损伤心脏和血管。提起和剪开心包膜时要细心,避免损伤心脏;
            \item 在心室射血后夹紧蛙心夹,防止弄破心室壁。记录收缩曲线时避免对心室的过度牵拉;
            \item 剪取心脏时.注意保留及心房和窦房结,使用温热任氏液(30$^{\circ}$C左右); 
          \end{itemize}
        }
        \item {
          \textbf{为什么没有观察到期外收缩和代偿间歇?}

          可能是因为心率过快,让心肌产生了不完全强直收缩,心脏本身没有充分的舒张,此时加入的额外刺激全部落在心脏收缩期和舒张早期的有效不应期,不产生期前收缩和代偿间歇。

          由于后面观察到的静脉窦搏动频率,和在体心脏的心率有接近30\% 的差异,同时,查阅相关资料发现,蟾蜍的平均心率约在45次/分钟上下,与观察到的60次并不一致,
          因此存在在体心脏心率过快的可能,可能导致上述所说的心肌不完全强直收缩,因而无法产生期前收缩和代偿间歇。

          在体心脏跳动过快的原因可能是受环境因素的影响,或是特定频率的电击所致,但是在计算心率前并没有经过电击,
          初步排除点击导致同步的可能性,故考虑当日的实验环境因素,或使蛙产生应激反应,使心率加快。
        }
        \item {
          \textbf{心肌和骨骼肌的区别及其神经机制?}

          节律性:心肌的兴奋源是心脏自身的窦房结,窦房结有节律地产生兴奋而引起心肌收缩。而骨骼肌收缩的兴奋源是运动神经中枢。

          钙离子依赖性:心肌细胞贮存的钙离子量比骨骼肌少,因此,心肌兴奋—收缩耦联所需的钙离子有一部分需要依赖于细胞外液中的钙离子内流。
          骨骼肌则由于肌质网贮备有大量的钙离子,因此受细胞外钙离子浓度改变影响较小。

          不应期:心肌细胞的有效不应期特别长,一直延续到心肌收缩活动的舒张早期,因此不会发生骨骼肌那样的完全强直收缩。

          同步收缩:心肌由于低电阻闰盘的存在,兴奋能在细胞间迅速传递,兴奋传至心房或心室时,几乎同时遍及整个心房或心室肌细胞,从而引起所有心房肌或心室肌同时收缩,称为“全或无”式收缩。
          而骨骼肌产生的兴奋不能在细胞之间直接传递,其同步收缩只能通过不同运动神经元和神经末梢同时发放神经冲动来引发,由于各神经元的兴奋性高低各不相同,所以其同步收缩性较差。

          骨骼肌神经机制:运动神经末梢动作电位——接头前膜去计划——电压门控钙通道开放——钙离子进入运动神经末梢——乙酰胆碱释放——终板膜对钠离子、钾离子通透性增高
          ——终板膜去极化——激活电压门控钠通道——骨骼肌细胞动作电位.

          心肌神经机制:窦房结兴奋——优势传导通路——房室交界——房室束及其左、右束支——普肯野纤维——心室肌——心室.
        }
        \item {
          \textbf{心肌期前收缩是否一定会伴随着代偿间歇的发生?}

          不一定,在窦性心律较慢时,再一次窦房结的兴奋也可在期前兴奋的有效不应期结束后才传到心室,在这种情况下,代偿间歇将不会出现。
        }
        \item {
          \textbf{超速驱动压抑及其神经机制}

          当自律细胞在受到高于其固有频率的刺激时,便按外加刺激的频率发生兴奋,称为超速驱动。在外来的超速驱动刺激停止后,自律细胞不能立即呈现其固有的自律性活动,
          需经一段静止期后才逐渐恢复其自身的自律性活动,这种现象称为超速驱动压抑(over-drive suppression)。由于窦房结的自律性远高于其他潜在起搏点,
          故窦房结的活动对潜在起搏点自律性的直接抑制作用就是一种超速驱动压抑。超速驱动压抑具有频率依赖性,即超速驱动压抑的程度与两个起搏点自动兴奋频率的差值相关,
          频率差值愈大,压抑效应愈强,驱动中断后,停止活动的时间也愈长。临床上常见的突然发生的窦性停搏时,往往要间隔较长时间才出现交界(房室结)性或室性的自主心律,
          就是这个道理。发生超速驱动压抑的原因之一是心肌细胞膜中钠泵活动的增强。当自律细胞受到超速驱动时,
          由于单位时间内产生的动作电位数目远超过按其自身节律所产生的动作电位数目,致使Na内流和K外流均增加,于是钠泵活动增强,同时外向性泵电流增大,
          使细胞膜发生超极化(即最大复极电位增大),因此自律性降低。当超速驱动压抑停止后,增强的钠泵活动并不立即恢复正常,故膜电位仍保持在超极化状态,
          此时该自律细胞自身4期自动去极化仍不能达到阈电位水平,故而出现一个短暂的心搏暂停时间,须待其自身的电活动恢复后,方可发生起搏活动。
        }
    \end{enumerate}
    \begin{thebibliography}{99}  
        \bibitem{ref1}于艳艳.温度对蟾蜍离体心脏心率及心率变异性的影响[D].中国医科大学,2008
        \bibitem{ref2}李秀国,李鸿雁,李海东,金文,孟凡,程中姣,庞淼.温度和湿度对蟾蜍生理功能的影响[J].四川动物,2013,(5): 734-738
    \end{thebibliography}
\end{document}