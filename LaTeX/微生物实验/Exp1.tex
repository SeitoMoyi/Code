\documentclass[UTF8]{article}
\usepackage{ctex,geometry,graphicx,float,makecell,rotating,multirow,diagbox}
\geometry{a4paper,scale=0.8}
\title{实验一\ 微生物接种技术}
\author{1组\ 徐培宾\ 201850050}
\date{2022年11月10日}

\begin{document}
    \maketitle
    \section{实验目的}
    \begin{enumerate}
    \item 学习使用常用的试验器具。
    \item 学习微生物接种的灭菌操作。
    \item 学习无菌操作。
    \end{enumerate}
    % \section{实验原理}
    % \subsection{蛙类心脏}
    % 蛙类心脏有两个心房和一个心室,心脏活动具有自律性,蛙心脏的起搏点是静脉窦。心房、心室的自律细胞称为潜在起搏点。静脉窦的节律最高,心房次之,心室最低。
    % 正常情况下,心脏的活动节律受控于静脉窦的节律,其活动顺序为:静脉窦、心房、心室。如果静脉窦的兴奋传导受阻,潜在起搏点可取代静脉窦引发心房、心室收缩活动。
    % 静脉窦、心房、心室的收缩活动可以通过张力传感器描记下来。

    % 自律细胞的兴奋性和节律与许多因素有关,其中体温是影响兴奋性的重要因素。分别单独加温心室、心房、静脉窦,观察其节律变化,可以证明三者节律性的高低。

    % 心肌的技能之一是具有较长的不应期,绝对不应期几乎占整个收缩期,有效不应期几乎相当于心室整个收缩期及舒张期的前三分之一。在心室收缩期内给予任何刺激,心室都不发生反应。
    % 在心室的舒张期特别是舒张的中后期,给予心室一个闽上刺激,则引发一次正常自律性以外的收缩反应,成为期前收缩或期外收缩。
    % 期外收缩本身也有一个有效不应期,当静脉窦传来的正常节律性兴奋刚好落在这个不应期内时,心室不再发生收缩反应,须待静脉窦传来下一次兴奋才能发生收缩反应。
    % 因此,在期外收缩之后,就会出现一个较长于正常节律性活动的间歇期,称为代偿间歇。

    % 心肌组织的这个特点,使得心肌不会像骨骼肌那样产生完全强直收缩,保证心肌始终作收缩与舒张的节律性活动,从而使心脏有血液回心充盈的时期,实现心脏的泵血功能。
    % \begin{figure}[H]
    %   \centering
    %   \includegraphics[width = 12cm]{heart.png}
    %   \caption{蛙类心脏}
    % \end{figure}
    % \subsection{心脏的容积导体特征}
    % 由于机体任何组织与器官都处于组织液的包围之中,而组织液作为导电性能良好的容积导体,可将组织和器官活动时所产生的生物电变化传至体表。
    % 故在体表或容积导体中的远隔部位可记录出某一器官或组织活动的电变化,如心脏活动所产生的生物电变化,可通过引导电极置于体表的不同部位记录下来,即心电图。
    \section{实验器材}
    接种针、接种环、接种钩、玻璃涂棒、接种圈、接种锄、小解剖刀
    % \begin{itemize}
    %     \item 用具:PowerLab实验系统、常用手术器械、培养皿、蛙心夹、蛙钉、蛙板、棉线、张力换能器、插入电极、滴管、培养皿、小试管、冰块、热水、支架、秒表、木夹等
    %     \item 材料:蛙
    %     \item 药品:任氏液
    % \end{itemize}
    \section{实验操作}
    \subsection{器具灭菌常用方法}
    \begin{itemize}
        \item 干热灭菌: 160~170℃干燥环境下,灭菌1-2h
        \item 高压蒸气灭菌:121 ℃环境下,蒸气灭菌15-30min 
        \item 灼烧灭菌
        \item 其它
    \end{itemize}
    \subsection{无菌操作}
    \begin{itemize}
        \item 无菌操作在超净工作台中进行
        \item 细菌在厌氧培养箱中进行无菌培养
        \item 细菌的接种和分离都需要在无菌条件下进行
    \end{itemize}
    \subsection{配制培养基}
    \subsubsection{培养基种类}
    \begin{itemize}
        \item 细菌:LB
        \item 真菌:PDA
        \item 放线菌:高氏培养基
    \end{itemize}
    \subsubsection{培养基配制}
    称量——加入三角瓶等——溶解——灭菌
    \subsubsection{倒平板}
    待配制好的培养基冷却至50℃左右后,按无菌操作法倒平板(每皿约倒15ml),平置,待凝。
    操作方法:右手持盛培养基的三角瓶置火焰旁边,用右手手掌和小指将瓶塞轻轻地拨出,
    瓶口保持对着火焰:然后左手拿接养并将皿盖在火焰附近打开一条稍大于瓶口的缝隙,迅速倒尺培养基约15ml,
    加盖后轻轻摇动培养皿,使培养基均匀分布在培养皿底部, 然后平置于桌面上,待凝后即为平板。
    % \begin{figure}[H]
    %   \centering
    %   \includegraphics[width = 12cm]{helec.png}
    %   \caption{心肌细胞动作电位与心电图的关系}
    % \end{figure}
    \subsubsection{三点接种}
    三点接种是为了获得单菌落所使用的方法之一,它是在培养皿上接种,即用接种针蘸取少量霉菌孢子,
    在浇有琼脂培养基的培养皿(俗称平板)上,以等边三角形的三点轻轻点一点,培养后即在此三点上长出菌落。
    其优点是:同种菌落有三个重复,同时在菌落彼此相接近的边缘常留有一条狭窄的空白地带此处菌丝生长稀疏,较透明,
    还分化出典型子实体,因此可以直接把培养皿放在低倍镜下观察,便于根据形态特点进行菌种的鉴定。具体操作如下:
    \begin{enumerate}
      \item 标明三点位置,待平板凝固后,在培养皿底部注明菌种、日期等,并以等边三角形的三个顶点标上记号。
      \item 右手拿接种针,先在火焰上烧红灭菌并在平板培养基的边缘冷却且蘸湿。
      \item 蘸取孢子,将灭过菌而且蘸湿的接种针伸入菌种管,用针尖蘸取少量霉菌孢子。
      \item 点接,以垂直法或水平法把接种针上的孢子点到预先标记好的部位,注意切勿刺破培养基。
    \end{enumerate}
    \subsubsection{划线接种}
    \begin{enumerate}
      \item 标记,在培养皿底面,用记号笔注明接种的菌名、接种者姓名。日期等。
      \item 灭菌接种环,点燃酒精灯,右手以持笔式握持接种环,并放置火焰中烧灼灭菌,先将接种环的接种丝部分置于火焰中,待金属丝烧红并蔓延至金属端,再直接烧灼金属环直至烧红,然后由金属环至金属杆方向快速通过火焰,随后再反方向通过火焰,如此2~3次。然后将接种环移开火焰,待其冷却。
      \item 取菌种,左手持装有金黄色葡萄球菌或大肠杆菌菌液的试管,用持有接种环的右手手掌及小指拔取试管塞, 将试管管口迅速通过火焰2~3次进行灭菌。 将已灭菌且己冷却的接种环伸入菌种管中,取接种环的菌液,然后退出菌种管,将菌种管管口再次通过火焰2~3次灭菌,塞好试管塞,放至原来的位置。
      \item {分离划线接种细菌(四区接种法),
      \begin{enumerate}
        \item 左手持琼脂平板培养基(将皿盖反放在桌上酒精灯附近),尽量使之直立以免空气中的细菌落入其中,并靠近火焰。右手持接种环在琼脂平板上端来回划线,涂成一细菌薄膜(约占平板的1/10), 视为一区。划线时使接种环与接种平板面呈30~40 度角,以腕力在平板表面行轻而快地来回滑动动作。
        \item 旋转琼脂平板90度,烧灼接种环,以杀灭环上的残留细菌,将接种环触及培养基表面以使其冷却灭菌接种环通过薄膜处作连续平行划线(约占平板 1/5),此视为二区。
        \item 转琼脂平板90度,灼烧接种环灭菌并使之冷却。将灭菌接种环接三区连续平行划线(约占平板 1/4),此为三区。
        \item 旋转琼脂平板90度,接种环不必再灭菌,接三区连续平行划线,划满平板其余部分,此视为四区。
      \end{enumerate}
      各区接种线间尽量互不交接,以达到细菌逐渐稀释的目的。
        }
    \end{enumerate}
    \section{注意事项}
    \begin{itemize}
        \item 倒平板:控制pH、灭菌时间、及时倒入培养基、注意标记。
        \item 划线接种:菌的数量要少、划线要轻。
    \end{itemize}
    % \begin{thebibliography}{99}  
    %     \bibitem{ref1}于艳艳.温度对蟾蜍离体心脏心率及心率变异性的影响[D].中国医科大学,2008
    %     \bibitem{ref2}李秀国,李鸿雁,李海东,金文,孟凡,程中姣,庞淼.温度和湿度对蟾蜍生理功能的影响[J].四川动物,2013,(5): 734-738
    % \end{thebibliography}
\end{document}